\documentclass{article}

\usepackage[utf8]{inputenc} 
\usepackage{amsmath}
\usepackage{graphicx}
\usepackage{listings}
\usepackage{color}

\title{Reporte Técnico}
\author{
  León Villapún, Luis Alfredo\\
  \texttt{A01322275}
  \and
  Canto García, Armando\\
  \texttt{A01322361}
  \and
  Rodiles Legaspi, Ricardo\\
  \texttt{A01325081}
  \and
  Tovar Muñoz de Cote, Alejandro\\
  \texttt{A01328484}
}

\begin{document}
	\pagenumbering{gobble}
	\maketitle
	\newpage
	\pagenumbering{arabic}
	
	\tableofcontents
	\newpage
	
	\section{Introducción}
	Dentro del marco del curso de Laboratorio de Desarrollo Web del ITESM Campus Puebla, se nos requirió hacer como proyecto final una plataforma web con temática a elegir.
	\linebreak
	La temática elegida para el proyecto fue una página de viajes. La idea en general es la de tener una página que fomente el turismo en Puebla y se puedan organizar viajes.
	\linebreak
	Así, se concibió un proyecto viable e interesante, que cumple con los objetivos del curso.
			
			
	\section{Github}
	La liga para el repositorio de github es la siguiente: https://github.com/Action52/PueblaViajes .
	\linebreak
	Aquí realizamos la organización de acuerdo a lo establecido por el forking model.
	
	
	\section{Trello}

	
	
	\section{Database}
	La base de datos fue montada en AWS con la ayuda de su funcionalidad RDS, para ser más específicos, como una base de Postgres.
	Se puede acceder a la base de datos con las siguientes credenciales:
	\linebreak
	Usuario: viajesPuebla
	\linebreak
	Password: viajes123
	\linebreak
	Puerto: 5432
	\linebreak
	Host: dbviajes.c72vt9uachn0.us-east-2.rds.amazonaws.com
	\linebreak
	Nombre de la base: viajesPuebla
	
	
	\section{REST API}
	Contamos con varios endpoints para que la aplicación funcione adecuadamente.
	\linebreak
	/viajes/ despliega todos los viajes o crea uno nuevo, dependiendo de la petición.
	\linebreak
	/viajes/id/ despliega la información de uno en específico, lo modifica o lo borra, dependiendo de la petición.
	\linebreak
	/users/ despliega todos los usuarios o crea uno nuevo, dependiendo de la petición.
	\linebreak
	/users/id/ despliega la información de uno en específico, lo modifica o lo borra, dependiendo de la petición.
	
	
	\section{Frontend Docker}
	
	
	\section{Frontend Testing}
	

	\section{Backend Testing}
	El testing del Backend se desarrolló con la ayuda que trae integrada Django. Se desarrollaron diversos test. Para probarlos simplemente se necesita correr el manage.py con la instrucción test.
	
	\section{Jenkins}
	Jenkins fue integrado de manera exitosa en la instancia de nuestro proyecto. Para acceder al mismo: http://18.217.0.33:8080/
	
	
\end{document}